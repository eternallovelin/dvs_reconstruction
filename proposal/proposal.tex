% ETH Zurich  - 3D Photography 2015
% http://www.cvg.ethz.ch/teaching/3dphoto/
% Template for project proposals

\documentclass[11pt,a4paper,oneside,onecolumn]{IEEEtran}
\usepackage{graphicx}
\usepackage{verbatimbox}
\usepackage[table]{xcolor}
\usepackage{booktabs}
% Enter the project title and your project supervisor here
\newcommand{\ProjectTitle}{Gradient reconstruction from DVS sensor}
\newcommand{\ProjectSupervisor}{Petri Tanskanen}
\newcommand{\DateOfReport}{March 6, 2015}

% Enter the team members' names and path to their photos. Comment / uncomment
% related definitions if the number of members are different than 2.  Including
% photographs are optional. Photos are there to help us to evaluate your group
% more effectively. If you wish not to include your photos, please comment the
% following line.
%\newcommand{\PutPhotos}{}
% Please include a clear photo of each member. (use pdf or png files for Latex to embed them in the document well)
\newcommand{\memberone}{Samuel Bryner}
\newcommand{\memberonepicture}{pic1.png}
\newcommand{\membertwo}{Marcel Geppert}
\newcommand{\membertwopicture}{pic2.png}

%%%% DO NOT EDIT THE PART BELOW %%%%
\title{\ProjectTitle}
\author{3D Photography Project Proposal\\Supervised by: \ProjectSupervisor\\ \DateOfReport}
\begin{document}
\maketitle
\vspace{-1.5cm}\section*{Group Members}\vspace{0.3cm}
\begin{center}\begin{minipage}{\linewidth}\begin{center}
\begin{minipage}{3 cm}\begin{center}\memberone\ifdefined\PutPhotos\\\vspace{0.2cm}\includegraphics[height=3cm]{\memberonepicture}\fi\end{center}\end{minipage}
\ifdefined\membertwo\begin{minipage}{3 cm}\begin{center}\membertwo\ifdefined\PutPhotos\\\vspace{0.2cm}\includegraphics[height=3cm]{\membertwopicture}\fi\end{center}\end{minipage}\fi
\ifdefined\memberthree\begin{minipage}{3 cm}\begin{center}\memberthree\ifdefined\PutPhotos\\\vspace{0.2cm}\includegraphics[height=3cm]{\memberthreepicture}\fi\end{center}\end{minipage}\fi
\end{center}\end{minipage}\end{center}\vspace{0.3cm}
%%%% END OF PROTECTED LINES %%%%


%%%% BEGIN WRITING THE DOCUMENT HERE %%%%


\section{Description of the project}

A Dynamic Vision Sensor (DVS, also called an event camera) is a camera that
asynchronously outputs spikes for pixels where the sensed intensity change
exceeds a given threshold, in contrast to full images at a fixed frequency like
a normal camera.

Our goal is to implement the approach of \cite{DBLP:conf/bmvc/KimHBID14} to
reconstruct a gradient map and finally a grayscale image of the scene from the
DVS signal.


\section{Work packages and timeline}

%Detailed descriptions of work packages you planned, their outcomes, the
%responsible group member and estimated timeline. Specify the challenges that
%will be tackled and considered solutions with possible alternatives, citing
%related documents if applicable. Mention the platform (Android, PC etc.) and
%the language (C++ etc.) you plan to use.

The approach in \cite{DBLP:conf/bmvc/KimHBID14} consists of three main parts:
tracking the camera motion, building a gradient map on basis of the estimated
camera rotation and the single pixel spikes, and computing a grayscale image
from this gradient map.

Our main challenge is likely the chicken-and-egg nature of the gradient map
building and tracking. The tracking of the camera rotation relies on the
existence of a gradient map while the construction of such a map requires an
estimation of the orientation to work. To handle this problem we will build a
simple simulation of the sensor output based on a given image. The simulation
will enable us to test the tracking algorithm without any mapping.

To test the construction of the gradient map independently, we will capture
test data where the orientation of the camera is obtained by an external
sensor.

The tracking of the camera rotation is done using a particle filter that tries
to estimate the camera rotation from the estimated movement of each single
pixel. The gradient map is built using a pixel-wise Kalman filter that
estimates the gradient of a pixel based on the estimated camera rotation.  For
the reconstruction of the grayscale image from the gradient map a Poisson image
reconstruction will be used. Here we can rely in large parts on publicly
available code what reduces the complexity of this task significantly.

The system will run on a PC using MATLAB. The project schedule is as follows:

\begin{center}
\rowcolors{1}{}{gray!25}
\addvbuffer[12pt]{\begin{tabular}{l l l}
Task & Responsible & Until\\
\toprule
\textbf{Proposal presentation} & & March 9\\
Simulation of camera output & Marcel & March 15\\
Capture test data with rotation & Samuel & March 15\\
Image reconstruction from gradient map & Marcel & March 22\\
Gradient map building & Samuel & April 10\\
Rotation tracking & Marcel & April 10\\
\textbf{Project updates} & & April 13\\
Component integration & Marcel & April 26\\
Refinement + tuning & Samuel & May 10\\
\textbf{Project demo} & & May 18
\end{tabular}}
\end{center}


\section{Outcomes and Demonstration}

%Give detailed information on the expected outcome of your project and the
%experiments you plan to test your implementation. If applicable, describe the
%online or offline demo you plan to present at the end of the semester.

We plan to reconstruct a grayscale image of the scene from the camera signal as
in \cite{DBLP:conf/bmvc/KimHBID14}. The evaluation of the method will be a
manual qualitative comparison of the generated images to the actual scene and
to images taken with a standard camera. For the demonstration we want to show a
standard picture of the scene together with an offline visualization of the raw
camera signal and reconstruction of the image from this signal, or possibly an
online reconstruction of the scene directly from the camera signal.


\vspace{1cm}
\textbf{Instructions:}

%\begin{itemize}
%\item The document should not exceed two pages including the references.
%\item Please name the document
%    \textbf{3DPhoto\_Proposal\_Surname1\_Surname2.pdf} and send it to
%    Ya\u{g}{\i}z in an email titled \textbf{[3DPhoto] Project Proposal -
%    Surname1 Surname2}, filling in your surnames.
%\end{itemize}

{%\singlespace
{\small
\bibliography{refs}
\bibliographystyle{plain}}}




\end{document}
