One of the central problems in robotics and augmented reality is localization:
Knowing where the robot or the user is. To be of any practical use,
localization must be fast and accurate, while relying on few or none external
means.

Vision based systems have long been attractive, for they provide very rich data
with reasonable speed while being very cheap. The drawback with normal cameras
is that high framerates are required to overcome motion blur resulting in
massive amounts of data. This in turn requires a lot of processing power to
extract the relevant information.

There is a new type of digital camera which overcomes these limitations by
closely mimicing a biological retina: A so called dynamic vision sensor (DVS),
which only outputs changes in the sensed image.

In this report, we investigate how such a camera (which is more closely
described in the next section \ref{sec:dvs}) can be used for basic visual SLAM.

This work is heavily based on \cite{kim2014simultaneous} and essentialy shares
the general structure.
