One of the central problems in robotics and augmented reality is localization:
Knowing where the robot or the user is. To be of any practical use,
localization must be fast and accurate, while relying on few or none external
means.
Vision based systems have been attractive for a long time, for they provide very rich data
with reasonable speed while being very cheap. The drawback with standard cameras
is that high framerates are required to overcome motion blur, resulting in
massive amounts of data. Additionally, the amount of useful information that can be extracted
from standard camera image can rapidly decrease with difficult, and especially changing
illumination conditions. This leads to the situation that, in many cases, 
a lot of processing power is required to extract the relevant information, or,
in a worst case, no relevant information can be extracted at all.
There is a new type of digital camera which overcomes these limitations by
closely mimicing a biological retina: A so called Dynamic Vision Sensor (DVS),
which only outputs changes in the sensed image independently for each pixel.
In this report, we investigate how such a camera (which is more closely
described in section \ref{sec:dvs}) can be used for basic visual SLAM.
This work is heavily based on \cite{kim2014simultaneous} and essentialy shares
the general structure.
