Similar to most SLAM algorithms, our algorithm consists of two co-dependent
steps. In the first step we try to estimate the current orientation based on
the camera signal, our map of the environment and the previous orientation.
This is then used to write the new data to the map and reconstruct a grayscale
image. This iteration is performed every time an event arrives from the camera.
Both the tracking and the rotation work on the assumption that the output  of
the other one is correct.

We assume a static scene, so every event (every change in brightness) is caused
by camera rotation or is due to noise.

Further, we only track camera rotation; We assume there is no translation at
all and therefore no parallax displacment either. This greatyl simplifies our
algorithm as we only have to keep track of a two dimensional map.

To speed up things even further, we assume that the camera's initial field
of view is already known and included in the map. With this assumption we do
not have to deal with a special initialization procedure and can immediately
start with the standard iteration while being relatively sure that the camera
movement is tracked correctly.

This is not an unreasonable assumption, as never DVS models incorporate the
ability to take full-frame pictures and a simple method for generating full
images can be used with any model (see Section about Experiments [TODO:
reference]).