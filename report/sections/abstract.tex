Inspired by biological eyes, a dynamic vision sensor (DVS) is a digital camera
which does not have a global or rolling shutter but where every pixel is
completely independent of each other and instantaneously reports changes in
sensed intensity.
Such a sensor provides much lower data bandwith, very high dynamic range and
extremely low latencies and is therefore very appealing for a wide range of
applications. However, due to the different nature of the data, standard
computer vision algorithms cannot be used directly.
In this report, we further investigate the approach proposed by Kim et al in
\cite{kim2014simultaneous}: DVS events are used to track the rotational pose of
a camera while simultaneously reconstructing a panoramic map of the
environment. Our work shows the general feasibility of this approach, but
suffers from slow performance.
