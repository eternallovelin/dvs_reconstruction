In order to integrate intensity change events into a full gradient map of the
environment, we must track the current position and movement direction of the
camera.

The camera's position is represented using a particle filter, where each
particle consists only of a weight and the three Euler angles necessary to
describe the camera's orientation.

Whenever a new event is received, we disturb the particles randomly with
variance proportional to the time since the last event. This is essentially a
constant position model, where the camera is assumed to stay stationary between
events but with uncertainty growing with time.

To update the weights of the disturbed particles, we retrieve the position of
the camera at the time of the last event of the *same* pixel (which can be a
lot earlier than the previous event). We then sample our map (from the
reconstruction part and which we assume is essentially correct) at this earlier
position. This intensity is then compared with the intensities at all the
proposed current positions: The closer the intensity difference to the
intensity threshold that generates an event, the liklier is the new position
and the more weight this particle gets.
