A Dynamic Vision Sensor (DVS) \cite{lpd08dvs}, also known as an event camera,
is a new kind of camera. In contrast to a standard camera it does not output a
complete image of the scene, but a series of events. An event is generated when
the sensed brightness of a pixel changes more than a certain threshold. This
happens for each pixel completely independent from all the others.

A DVS has several advantages compared to standard cameras. Since there is no
need to gather data from all pixels to generate an image, events can be sent
with an extremely low latency of 15 $\mu$s or less \cite{lpd08dvs, brandli14davis}.
For the same reason, there is practically no motion blur in the DVS signal.
Another benefit of independent pixels is the very high dynamic range.
Saturation, blooming or smearing as observed when a bright light source is
captured with a standard camera is basically nonexistent. Finally, due to the
reduction of the camera signal to changes and the resulting elimination of
redundant information, the signal bandwith is much smaller than with a standard
camera.

All these features make the DVS a very desirable sensor for motion tracking,
especially for mobile robots where the data has to be analyzed on constrained
hardware.

However, there are currently several caveats: First of all, currently available
DVS (DVS128 \cite{lpd08dvs}, DAVIS \cite{brandli14davis}) have a very limited
resolution of 128 $\times$ 128 or 240 $\times$ 180 pixels, respectively. This
obviously limits the spacial accuracy of the sensed data, but it is likely to
be a simple matter of time until sensors with a higher resolution are
developed. A more important point is that the computer vision algorithms that
have been developed during the last decades can either not be used at all or
have to be adapted to the new data representation.
