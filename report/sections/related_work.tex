Computer Vision using Dynamic Vision Sensors is a very young field since
the sensors have only been around for a few years \cite{lpd08dvs, brandli14davis}.
While the problem of visual odometry and visual SLAM has been known for many years
and many approaches have been proposed, the work by Kim \etal \cite{kim2014simultaneous},
which is the basis of our work, is, to our best knowledge, the first approach to the problem
using only a DVS with no additional information.
It has, however, already been shown that the precision of a visual odometry system using
a standard camera can be increased by adding a DVS,
especially during fast turns when the standard camera suffers from motion blur \cite{censi13dvsd_sub}.
Additionally, DVS's have been shown to outperform standard cameras at pose tracking during high-speed maneuvers
when used in sufficiently equipped environments \cite{mueggler2014event}.
Other research with DVS's focuses in great parts on tracking moving objects
with a static camera \cite{vmv.20141280, conradt2009embedded, censi13led}.