The goal of this project was to implement full rotational tracking and scene
reconstruction using only a DVS. We implemented an approach described by Kim \etal
in \cite{kim2014simultaneous} using Matlab. It uses a SLAM-like two-step
algorithm, where the position and the map are jointly estimated and is
described in Section \ref{sec:core_algorithm}.

In contrast to \cite{kim2014simultaneous} we are working with some
simplifications: The particle filter for the tracking directly uses Euler
angles and corner cases like gimbal lock are not handled.

As we discovered during the implementation, the runtime of the algorithm is
extremely long when implemented in Matlab. The events that are created by the
camera in only a few seconds can easily lead to hours of computation.

To still be able to test the algorithm and in order to facilitate development
we implemented and exclusively used a full DVS simulator that generates events
from a spherical panorama.  This allows us to develop and test each part of the
algorithm independently with perfect input data and ground truth. It also makes
experimentation easier and faster as everything can be inspected at will and we can
control the number of events by tweaking parameters of our virtual camera.

To speed things up, we assume that the camera's initial field
of view is already known and included in the map. With this assumption we do
not have to deal with a special initialization procedure and can immediately
start with the standard iteration while being relatively sure that the camera
movement is tracked correctly.

This is not an unreasonable assumption, as newer DVS models incorporate the
ability to take full-frame pictures and there's a simple method for generating
full images which works with any hardware model (see section
\ref{sec:shutter_removal}).

As for the work distribution in our group: we implemented everything ourselves
(apart from the poisson image reconstruction from image gradients which we took
from \cite{raskarpoisson}). Most of the code was developed jointly, with Marcel
Geppert focusing on the image reconstruction part and Samuel Bryner on the
tracking part.
